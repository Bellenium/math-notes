\chapter{Vector Spaces}
\section{Fields}
A field $\mathcal{F}$ is an abstract algebraic object. Throughout these notes $\mathcal{F}$ stands for either $\mathbb{R}$ or $\mathbb{C}$. \footnote{Many of theorems and definitions work even if replace $\mathcal{F}$ with an arbitrary field.}
\section{Complex Numbers}
A complex number is an order pair ${} \in \mathbb{C}$ where $a,b \in \mathbb{R}$ where we can denote it as $z = a + ib$ where $i = \sqrt{-1}$
\subsection{Addition}
$z_{1} = a_{1} + ib_{1}, \ z_{2} = a_{2} + ib_{2}$
$$z_{1} + z_{2} =  (a_{1} + a_{2}) + i(b_{1} + b_{2})$$
\subsection{Multiplication}
$z_{1} = a_{1} + ib_{1}, \ z_{2} = a_{2} + ib_{2}$
$$z_{1}z_{2} =  (a_{1} + ib_{1})(a_{2} + ib_{2}) = (a_{1}a_{2} - b_{1}b_{2}) + i(a_{1}b_{2} + a_{2}b_{1})$$
\subsection{Properties}\footnote{$\mathcal{W}, \mathcal{Z}, \lambda \in \mathbb{C}$}
\subsubsection{Commutativity}
$$\mathcal{W} + \mathcal{Z} = \mathcal{Z} + \mathcal{W}$$
$$\mathcal{W}\mathcal{Z} = \mathcal{Z}\mathcal{W}$$
\subsubsection{Associativity}
$$(\mathcal{Z}_1 + \mathcal{Z}_2) + \mathcal{Z}_3 = \mathcal{Z}_1 + (\mathcal{Z}_2 + \mathcal{Z}_3)$$
$$(\mathcal{Z}_1\mathcal{Z}_2)\mathcal{Z}_3 = \mathcal{Z}_1(\mathcal{Z}_2\mathcal{Z}_3)$$
\subsubsection{Identities}
$$\mathcal{Z} + 0 = \mathcal{Z}$$
$$\mathcal{Z}1 = \mathcal{Z}$$
\subsubsection{Additive Inverse}
$$\forall \ \mathcal{Z} \ \exists \ \mathcal{Z}^{-1} \ | \ \mathcal{Z} + \mathcal{Z}^{-1} = 0$$
\subsubsection{Multiplicative Inverse}
$$\forall \  \mathcal{Z} \neq 0 \ \exists \ \mathcal{W} \ | \ \mathcal{Z}\mathcal{W} = 1$$
\subsubsection{Distributive Property}
$$\lambda(\mathcal{W} + \mathcal{Z}) = \lambda\mathcal{W} + \lambda\mathcal{Z}$$
\section{Notation}
\textit{\textbf{n-tuple}} refers to an ordered set of $n$ numbers over a field $\mathcal{F}$.
\section{Definition of a Vector Space}
\label{sec: Sec 1.3}
A vector space $\mathbb{V}$ is a set along with the regular multiplication and addition operations over a field $\mathcal{F}$, such that the following axioms hold: \footnote{Here, $\alpha , \beta \in \mathcal{F}$ and $\mathcal{U}, \mathcal{V} $ and $\mathcal{W} \in \mathbb{V}$} \\
\subsection{Commutativity}
 $$\mathcal{U} + \mathcal{V} = \mathcal{V} + \mathcal{U}$$
\subsection{Associativity}
 $$(\mathcal{U} + \mathcal{V}) + \mathcal{W} = \mathcal{V} + (\mathcal{U} + \mathcal{W})$$
 $$(\alpha \beta) \mathcal{V} = \alpha (\beta \mathcal{V})$$
\subsection{Additive Identity}
$$\exists \  0 \in \mathbb{V} \ | \ \mathcal{V} + 0 = 0 + \mathcal{V} = \mathcal{V}$$
\subsection{Additive Inverse}
$$\forall \ \mathcal{V} \ \exists \ \mathcal{V}^{-1} \ | \ \mathcal{V} + \mathcal{V} = 0$$
\subsection{Multiplicative identity}
$$\exists \ 1 \in \mathbb{V} \ | \ 1 \mathcal{V} = \mathcal{V}$$
\subsection{Distributive properties}
$$\alpha (\mathcal{U} + \mathcal{V}) = \alpha \mathcal{U} + \alpha \mathcal{V}$$
$$(\alpha + \beta) \mathcal{U} = \alpha \mathcal{U} + \beta \mathcal{U}$$
\section{Properties of a Vector Space}
\subsection{\textit{A vector space has a unique additive identity}}
Suppose there exist two additive identities $0$ and $0^{'}$ for the vector space $\mathbb{V}$, we can say that
$$0 = 0 +  0^{'} = 0^{'}$$
Thus,
\begin{tcolorbox}
	\begin{equation}
		0 = 0^{'}
	\end{equation}
\end{tcolorbox}
\subsection{Ever element in a vector space has a unique additive inverse}
Suppose where $\mathcal{W}$ and $\mathcal{W}^{'}$ are the additive inverses of $\mathcal{V}$, then

\begin{tcolorbox}
\begin{equation}
\mathcal{W} = \mathcal{W}^{'}
\end{equation}
\end{tcolorbox}
 
\subsection{$0\mathcal{V} = 0 \  \forall \ \mathcal{V} \in \mathbb{V}$}
$\forall \ \mathcal{V} \in \mathbb{V}$,
$$0 \mathcal{V} = (0 + 0) \mathcal{V} = 0 \mathcal{V} + 0 \mathcal{V}$$
$$0 \mathcal{V} - 0 \mathcal{V} = 0 = 0 \mathcal{V} $$
Thus,
\begin{tcolorbox}
	\begin{equation}
		 0 = 0 \mathcal{V}
	\end{equation}
 \end{tcolorbox}
\subsection{$0\alpha = 0 \  \forall \ \alpha \in \mathcal{F}$}
$\forall \ \alpha \in \mathbb{F}$,
$$0 \alpha = (0 + 0) \alpha = 0 \alpha + 0 \alpha$$
$$0 \alpha - 0 \alpha = 0 = 0 \alpha $$
Thus,
\begin{tcolorbox}
	\begin{equation}
	0\alpha = 0
	\end{equation}
\end{tcolorbox}
\subsection{$(-1)\mathcal{V} = -\mathcal{V} \ \forall \ \mathcal{V} \in \mathcal{F}$}
$\forall \ \mathcal{V} \in \mathbb{V}$,
$$0 \mathcal{V} = (0 + 0) \mathcal{V} = 0 \mathcal{V} + 0 \mathcal{V}$$
$$0 \mathcal{V} - 0 \mathcal{V} = 0 = 0 \mathcal{V} $$
Thus,
\begin{tcolorbox}
	\begin{equation}
	0 = 0 \mathcal{V}
	\end{equation}
\end{tcolorbox}
\section{Subspaces}
\subsection{Definition}
A $\mathbb{U} \subset \mathbb{V}$ is called a \textit{\textbf{subspace}} of $\mathbb{V}$ if $\mathbb{U}$ is also a vector space as defined in \hyperref[sec: Sec 1.3]{Sec 1.3}
\subsection{Properties}
If $\mathbb{U} \subset \mathbb{V}$ then to check whether $\mathbb{U}$ is a subspace of $\mathbb{V}$, we simply need to check for the following properties
\subsubsection{Additive identity}
$$0 \in \mathbb{U}$$
\subsubsection{Closed under addition}
$$\mathcal{U}, \mathcal{V} \in \mathbb{U} \implies \mathcal{U} + \mathcal{V} \in  \mathbb{U}$$
\subsubsection{Closed under scalar multiplication}
$$\forall \  \alpha \in \mathcal{F} \ and \ \mathcal{U} \in \mathbb{U} \implies \alpha \mathcal{U} \in \mathbb{U}$$
\section{Sums}
The sum of $\mathcal{U}$ and $\mathcal{V}$ which are subspaces of $\mathbb{V}$ is defined to be the set of all poissible sums of the elements is denoted in the RHS as,
$$\mathcal{U} + \mathcal{V} = \{u + v: u \in \mathcal{U}, v \in  \mathcal{V}\}$$

\section{Direct Sums}
A direct sum of sub-spaces is a special type of sum in which
\subsection{Proposition 1}
Suppose $\mathbb{U}_1, \mathbb{U}_2$ are subspaces of $\mathbb{V}$. Then $\mathbb{V} = \mathbb{U}_1 \oplus \mathbb{U}_2$ if and only if both the following conditions hold:
\begin{itemize}
	\item $\mathbb{V} = \mathbb{U}_1 + \mathbb{U}_2$
	\item the only wayt to write - as a sum $\mathcal{U}_1 + \mathcal{U}_2$, where each $\mathcal{U}_j \in \mathbb{U}_j$, is by taking all the $\mathcal{U}_j = 0 $
\end{itemize}
\subsubsection{Proof}
 First suppose that $\mathbb{V} = \mathbb{U}_1 \oplus \mathbb{U}_2$. Clearly the first condition holds because of how sum and direct sum are defined. To prove the latter suppose $\mathcal{U}_1 \in \mathbb{U}_1, \mathcal{U}_2 \in \mathbb{U}_2$ and
 $$0 = \mathcal{U}_1 + \mathcal{U}_2$$
 Then each $\mathcal{U}_i$ must be, as this follows from the uniqueness part of the definition of direct sum because $0 = 0 + 0$ and $0 \in \mathbb{U}_1, 0 \in \mathbb{U}_2$. Now suppose that both the conditions hold. Let $\mathcal{V} \in \mathbb{V}$. By the first condition we can write:
 $$\mathcal{V} =  \mathcal{U}_1 + \mathcal{U}_1$$
 for some $\mathcal{U}_1 \in \mathbb{U}_1$ and $\mathcal{U}_2 \in \mathbb{U}_2$. To show that this representation is unique, suppose we also have:
 $$\mathcal{V} = \mathcal{V}_1 + \mathcal{V}_2$$
 where $\mathcal{V}_1 \in \mathbb{U}_1$ and $\mathcal{V}_2 \in \mathbb{U}_2$. Subtracting these two equations we have
 $$0 = (\mathcal{U}_1 - \mathcal{V}_1) + (\mathcal{U}_2 - \mathcal{V}_2)$$
 Clearly $\mathcal{U}_i - \mathcal{V}_i \in \mathbb{U}_i$, so the equation above and the second condition imply that each $\mathcal{U}_i - \mathcal{V}_i = 0$. Thus, $\mathcal{U}_i = \mathcal{V}_i$ as desired.
\subsection{Proposition 2}
Suppose that $\mathbb{U}$ and $\mathbb{W}$ are subspaces of $\mathbb{V}$. Then $\mathbb{V} = \mathbb{U} + \mathbb{W} \ $ i.f.f. $\mathbb{V} = \mathbb{U} + \mathbb{W}$ and $\mathbb{V} \cap \mathbb{W} = 0$.
\subsubsection{Proof}
First suppose that $\mathbb{V} = \mathbb{U} \oplus \mathbb{W}$. Then $\mathbb{V} = \mathbb{U} + \mathbb{W}$, by the definition of a direct sum. Also, if $\mathcal{V} \in \mathbb{U} \cap \mathbb{W}$, then $0 = \mathcal{V} + (-\mathcal{V})$, where $\mathcal{V} \in \mathbb{U}$ and $-\mathcal{V} \in \mathbb{W}$. By the unique reppresentation of  as the sum of a vector $\mathbb{U}$ in and a vector in $\mathbb{W}$, we must have $\mathcal{V} = 0$.Thus, $\mathbb{U} \cap \mathbb{W} = \{0\}$ . This is one way to prove it. \\

To prove the other way, now suppose that $\mathbb{V} = \mathbb{U} + \mathbb{W}$ and $\mathbb{U} \cap \mathbb{W} = {0}$. To prove that $\mathbb{V} = \mathbb{U} \oplus \mathbb{W}$, suppose that 
$$0 = \mathcal{U} + \mathcal{V}$$
where $\mathcal{U} \in \mathbb{U}$ and $\mathcal{W} \in \mathbb{W}$. To complete the proof, wee only need to show that $\mathcal{U} = \mathcal{W} = 0$. The equation above implies that $\mathcal{U} = - \mathcal{W} \in \mathbb{W}.$ from axiom 4. Thus, $\mathcal{U} \in \mathbb{U} \cap \mathbb{W}$, and hence $\mathcal{U} = 0$
\section{Note}
\begin{itemize}
	\item Sums of subspaces are analogous to unions of subsets
	\item Similarly, direct sums of subspaces are analogous to disjoint unions of subset i.e. they have no element in common
	\item No two subspaces of a vector space can be disjoint because both must contain $0$ as per the axioms stated in 
	\item Thus, disjointness is replaced, at least in the case of two subspaces, with the requirement that the intersection equals ${0}$
\end{itemize}