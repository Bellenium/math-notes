\chapter{The First Chapter}
There is precisely one simply-connected Lie group corresponding
to each Lie algebra.
\section{Representation}
A representation $R$ of a group element $g$ is a one to one map to an element of a vector space i.e. it is homeomorphic.
$$g \rightarrow R(g)$$
The following properties are preserved:
\begin{itemize}
\item $R(e) = I$
\item $R(g^{-1}) = (R(g))^{-1}$
\item $R(g) \circ R(h) = R(gh)$
\end{itemize}
A representation identifies with each point (abstract group eement) of the group manifold (the abstract group) a linear transformation of a vector space. Ggenerally if one accepts arbitrary (not linear) transformations of an arbitrary
(not necessarily a vector) space. Such a map is called a realization. 
\subsection{Similartiy Transform}
$$R \rightarrow R^{'} := S^{-1}RS$$
This means that if we have a representation, we can transform its elements wildly with literally any non-singular matrix $S$ \footnote{$det(s) \neq 0$} to get nicer matrices.
\subsection{Invariant Subspaces}
This means, if we have a vector in the subspace $V'$ and we act on it with arbitrary group elements, the transformed
vector will always be again part of the subspace V'. If we find such an invariant subspace we can define a representation R' of G on V', called a subrepresentation of R, by
$$R'(g)v = R'(g)v$$
This means that the representation R is actually composed of smaller building blocks called subrepresentations.
\subsection{Irreducible Representation}
An irrep  is a rep of a Group $G$ on a vector space $V$ that has no invariant subspaces other than $\{ 0 \}$ and $V$ itself.\\
There are many possible representations for each group 86 , how do
we know which one to choose to describe nature? There is an idea
that is based on the Casimir elements. A Casimir element $C$ is built
from generators of the Lie algebra and its defining feature is that it
commutes with every generator $X$ of the group.
\begin{equation}
	[C, X ] = 0
\end{equation}
A famous Lemma, called Schur’s Lemma tells us that if we have an irreducible representation,
$R: \mathfrak{g} \rightarrow GL(V)$, any linear operator $T: V \rightarrow V$ that commutes with all operators $R(X)$ must be a scalar multiple of the identity operator. Therefore,
the Casimir elements give us linear operators with constant values for each representation. As we will see, these values provide us with
a way of labelling representations naturally. 88 We can therefore start
to investigate the irreducible representations, by starting with the
representation with the lowest possible scalar value for the Casimir
element.\\

An irreducible representation cannot be rewritten, using a similarity
transformation, in block diagonal form 84 . In contrast, a reducible
representation can be rewritten in block-diagonal form through sim-
ilarity transformations. These notions are important because we use
irreducible representations to describe elementary particles 85 . We will
see later that the behavior of elementary particles under transforma-
tions is described by irreducible representations of the corresponding
symmetry group.\\
 An important observation
that helps us to make sense of the vector space, is that for any Lie
group, one or several of the corresponding generators can be diag-
onalized using similarity transformations. In physics we use these
diagonal generators to get labels for the basis vectors that span our
vector space. We use the eigenvectors of these diagonal generators
as basis for our vector space and the corresponding eigenvalues as
labels. This idea is incredibly important to actually understand the
physical implications of a given group. If there is just one gener-
ator that can be diagonalized simultaneously, each basis vector is
labelled by just one number: the corresponding eigenvalue. If there
are several generators that can be diagonalized simultaneously, we
get several numbers as labels for each basic vector. Each such number
is simply the eigenvalue of a given diagonal generator that belongs
to this basic vector (=eigenvector). In particular this is where the
"charge labels" for elementary particles: electric charge, weak charge,
color charge, come from. 

\section{Poincare Group}
The Poincare Group = Lorentz Group + Translations in time and space
$$ = \mathbb{R}^{1,3} \rtimes ISO(1,3)^{\uparrow}$$



