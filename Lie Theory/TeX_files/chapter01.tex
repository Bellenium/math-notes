\chapter{The First Chapter}
\section{Representation}
A representation $R$ of a group element $g$ is a one to one map to an element of a vector space i.e. it is homeomorphic.
$$g \rightarrow R(g)$$
The following properties are preserved:
\begin{itemize}
\item $R(e) = I$
\item $R(g^{-1}) = (R(g))^{-1}$
\item $R(g) \circ R(h) = R(gh)$
\end{itemize}
A representation identifies with each point (abstract group eement) of the group manifold (the abstract group) a linear transformation of a vector space. Ggenerally if one accepts arbitrary (not linear) transformations of an arbitrary
(not necessarily a vector) space. Such a map is called a realization. 
\subsection{Similartiy Transform}
$$R \rightarrow R^{'} := S^{-1}RS$$
This means that if we have a representation, we can transform its elements wildly with literally any non-singular matrix $S$ \footnote{$det(s) \neq 0$} to get nicer matrices.
\subsection{Invariant Subspaces}
This means, if we have a vector in the subspace $V'$ and we act on it with arbitrary group elements, the transformed
vector will always be again part of the subspace V'. If we find such an invariant subspace we can define a representation R' of G on V', called a subrepresentation of R, by
$$R'(g)v = R'(g)v$$
This means that the representation R is actually composed of 
\subsection{Irreducible Representation}

